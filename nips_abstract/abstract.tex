%% LyX 1.6.7 created this file.  For more info, see http://www.lyx.org/.
%% Do not edit unless you really know what you are doing.
\documentclass[10pt,english]{article}
\usepackage{mathptmx}
\usepackage[T1]{fontenc}
\usepackage[latin9]{inputenc}
\setlength{\parskip}{\medskipamount}
\setlength{\parindent}{0pt}

\makeatletter
%%%%%%%%%%%%%%%%%%%%%%%%%%%%%% User specified LaTeX commands.
\usepackage{nips10submit_e,times}
\usepackage{microtype}
\nipsfinalcopy

\makeatother

\usepackage{babel}

\begin{document}

\title{A Markov-Chain Monte Carlo Approach to Simultaneous Localization
and Mapping}


\author{P�ter Torma\\
GusGus AB\\
Hungary\\
\And Andr�s Gy�rgy\\
Machine Learning Res. Group\\
MTA SZTAKI, Hungary\\
\And Csaba Szepesv�ri, Roshan Shariff\\
Dept. of Computing Sciences\\
University of Alberta, Canada}

\maketitle
\thispagestyle{empty}

Simultaneous Localization and Mapping (SLAM) is the intensively studied
problem of determining a mobile robot's position in an unknown environment,
while simultaneously building a consistent map of the environment.
The robot is assumed to be moving under a known control input, which
it follows with some uncertainty, in an environment containing landmarks
which it can observe. It is assumed that the control and observation
uncertainties can be modelled probabilistically.

These probabilistic dynamics and observation models give rise to a
posterior distribution over the robot's trajectory and landmark positions.
Assuming a prior distribution over the landmark positions, the theory
of Markov-chain Monte Carlo (MCMC) methods allows us to represent
the posterior as the stationary distribution of a Markov process.
We describe a provably consistent and rapidly converging algorithm
that samples from the SLAM posterior using MCMC methods for arbitrary
dynamics and observation models.

Previous SLAM algorithms either assume linearized models and Gaussian
noise (e.g. EKF-SLAM) or use particle filters (e.g. FastSLAM). Since
real-world robots usually have non-linear dynamics, the former approach
may fail to converge. Particle filters approximate the posterior distribution
over robot trajectories with a set of samples drawn from it. This
fails when the certainty in the robot's position increases sharply,
which happens when the robot observes a landmark it has seen earlier
({}``a loop is closed'') and is thus able to correct accumulated
errors in its position. If no sample in the set lies close to the
true trajectory, the particle set collapses to an inaccurate estimate.

The new algorithm constructs an \emph{inference graph} in which the
vertices represent the parameters to be estimated (i.e. robot and
landmark locations) and the edges represent the constraints between
them (i.e. the robot's movements and observations). A labeling of
this graph is an estimate of each of the unknown parameters and constraints,
and is said to be \emph{consistent} if all the parameter values satisfy
the constraints. The robot dynamics and observation models assign
a probability density to any labeling of the inference graph. By randomly
selecting and resampling an edge label from a spanning tree of the
graph, we are able to sample from a proposal distribution that maintains
the consistency property. We then use a generalized Metropolis-Hastings
algorithm to sample a consistent labeling, thus determining a map
and trajectory which are sampled, in the limit, from the SLAM posterior
distribution. Since the graph stores the entire history of observations
and controls, the edge resampling technique is able to correct errors
arbitrarily far back in time, thereby successfully tracking the robot's
position even after long loops have been closed.

We describe how the structure of the SLAM inference graph makes this
edge resampling algorithm efficient, and suggest implementation techniques.
We present results demonstrating that the resulting Markov process
converges rapidly to the correct map and trajectory estimates. MCMC
methods allow the algorithm to improve on previous techniques without
compromising correctness or losing convergence guarantees.

\bibliographystyle{plain}
\nocite{*}
\bibliography{abstract}

\end{document}
